\chapter*{Résumé}
\addcontentsline{toc}{chapter}{Résumé}

\renewcommand{\abstractnamefont}{\normalfont\Large\bfseries}
%\renewcommand{\abstracttextfont}{\normalfont\Huge}

%\begin{abstract}
\hskip7mm

\begin{spacing}{1.3}

\end{spacing}

Le virus du chikungunya, transmis principalement par les moustiques femelles \textit{Aedes aegypti} et \textit{Aedes albopictus}, représente une menace croissante pour la santé publique mondiale en raison de sa propagation rapide et de ses effets débilitants\footnote{affaiblissants}. Récemment, des épidémies ont été signalées non seulement en Afrique centrale et orientale, mais aussi en Amérique du Sud et en Asie du Sud-Est. Prédire ces épidémies reste un défi majeur en raison de l'interaction complexe entre les facteurs environnementaux, climatiques et biologiques. Les approches traditionnelles de surveillance épidémiologique se révèlent souvent insuffisantes pour anticiper les épidémies de manière proactive. Cette recherche se concentre sur le \textbf{Tchad}, le \textbf{Brésil} et le \textbf{Paraguay}, trois pays où le chikungunya a émergé comme un problème significatif de santé publique. En raison de la disponibilité limitée des données au Tchad, des données supplémentaires provenant du Brésil et du Paraguay sont intégrées pour renforcer l'analyse. Le but ici est de développer et d'évaluer des modèles prédictifs pour les épidémies de chikungunya en utilisant des techniques avancées d'apprentissage automatique, en particulier la régression d'ensemble. Les modèles choisis pour cette étude incluent le \textbf{Random Forest Regressor} et le \textbf{XGBoost Regressor} optimisé via \textbf{Grid Search}, ainsi qu'un modèle d'ensemble (\textbf{Voting Regressor}) combinant \textbf{Linear Regression}, \textbf{Random Forest Regressor} et le \textbf{XGBoost Regressor} optimisé. Ces modèles seront formés et validés à partir de données épidémiologiques et climatiques. Parmi ces modèles, notre modèle d'ensemble \textbf{Voting Regressor}, qui combine les prédictions des modèles \textbf{Linear Regression}, \textbf{Random Forest Regressor} et \textbf{XGBoost Regressor}, a affiché des performances globalement supérieures aux autres modèles individuels, avec un \textbf{MAE} minimal, un \textbf{RMSE} relativement bas et une très bonne précision (\textbf{65\%} pour le Brésil, \textbf{28,2\%} pour le Tchad et \textbf{59,57\%} pour le Paraguay). Les analyses épidémiologiques et statistiques ont révélé une corrélation entre les conditions climatiques et le nombre de cas liés au chikungunya.

\textbf{Mots-clés} : \textit{Chikungunya, Grid Search, Régression d'ensemble, Random Forest, XGBoost, Apprentissage Automatique, Voting Regressor}


%\end{abstract}
