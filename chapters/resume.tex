\chapter*{Résumé}
\addcontentsline{toc}{chapter}{Résumé}

\renewcommand{\abstractnamefont}{\normalfont\Large\bfseries}
%\renewcommand{\abstracttextfont}{\normalfont\Huge}

%\begin{abstract}
\hskip7mm

\begin{spacing}{1.3}

\end{spacing}

Le virus du chikungunya, transmis principalement par les moustiques \textbf{Aedes aegypti} et \textbf{Aedes albopictus}, représente une menace croissante pour la santé publique mondiale en raison de sa propagation rapide et de ses effets débilitants. Récemment, des épidémies ont été signalées non seulement en Afrique centrale et orientale, mais aussi en Amérique du Sud et en Asie du Sud-Est. Prédire ces épidémies reste un défi majeur en raison de l'interaction complexe entre les facteurs environnementaux, climatiques et biologiques. Les approches traditionnelles de surveillance épidémiologique se révèlent souvent insuffisantes pour anticiper les épidémies de manière proactive.

Cette recherche se concentre sur le \textbf{Tchad}, le \textbf{Brésil}, et le \textbf{Paraguay}, trois pays où le chikungunya a émergé comme un problème significatif de santé publique. En raison de la disponibilité limitée de données au Tchad, des données supplémentaires provenant du Brésil et du Paraguay sont intégrées pour renforcer l'analyse. Le but de cette thèse est de développer et d'évaluer des modèles prédictifs pour les épidémies de chikungunya en utilisant des techniques avancées d'apprentissage automatique, en particulier la régression d'ensemble.

Les modèles choisis pour cette étude incluent le \textbf{Random Forest Regressor}, le \textbf{XGBoost Regressor} optimisé via \textbf{Grid Search}, ainsi qu'un modèle d'ensemble (\textbf{Voting Regressor}) combinant \textbf{Linear Regression}, \textbf{Decision Tree Regressor}, et le \textbf{XGBoost Regressor} optimisé. Ces modèles seront formés et validés à partir de données épidémiologiques et climatiques.

Cette thèse commence par une analyse approfondie de l'épidémiologie du chikungunya et de ses dynamiques de transmission, suivie d'une revue des applications actuelles de l'intelligence artificielle dans la prédiction des maladies. Les modèles sont ensuite testés et évalués sur la base de critères de performance tels que la sensibilité, la spécificité, et la valeur prédictive.

Les résultats de cette recherche montrent l'efficacité des techniques de régression d'ensemble pour améliorer la précision des prévisions épidémiologiques, et offrent des perspectives nouvelles pour l'intervention en santé publique dans les régions affectées.

\textbf{Mots-clés} : \textit{Chikungunya, Régression d'ensemble, Random Forest, XGBoost, Apprentissage Automatique, Données Climatiques, Prévision Épidémiologique}

%\end{abstract}
