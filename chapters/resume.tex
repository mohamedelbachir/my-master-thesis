\chapter*{Résumé}
\addcontentsline{toc}{chapter}{Résumé}

\renewcommand{\abstractnamefont}{\normalfont\Large\bfseries}
%\renewcommand{\abstracttextfont}{\normalfont\Huge}

%\begin{abstract}
\hskip7mm

\begin{spacing}{1.3}

 

%Les insights obtenus ont des implications pour l'amélioration des systèmes d'alerte précoce et l'élaboration de stratégies ciblées de santé publique visant à atténuer l'impact des épidémies de chikungunya à l'échelle mondiale.

%\textbf{Mots clés}: \textit{Système de transport intelligent, Prédiction des vitesses de trafic, Apprentissage automatique, IoT.} 

\end{spacing}
Le virus du chikungunya, transmis par les moustiques \textbf{Aedes aegypti}  et \textbf{Aedes albopictus}, constitue une menace importante pour la santé publique mondiale en raison de sa propagation rapide et de son impact débilitant sur les populations touchées. Ces dernières années, des épidémies ont touché des populations d'Afrique centrale et orientale, d'Amérique du Sud et d'Asie du Sud-Est ~\cite{intro}. Prédire les épidémies de chikungunya est un défi, en raison de l'interaction complexe de facteurs environnementaux, sociaux et biologiques. Les méthodes traditionnelles de surveillance épidémiologique, bien qu'essentielles, peuvent s'avérer insuffisantes pour prévoir les épidémies de manière proactive. Cette recherche se concentre spécifiquement sur le \textbf{Tchad}, un pays où le chikungunya a récemment émergé comme un problème majeur de santé publique. Avec l'augmentation des cas de chikungunya au Tchad, il est crucial de développer des modèles prédictifs précis pour anticiper les épidémies et mettre en œuvre des mesures de contrôle efficaces. L'avènement de l'intelligence artificielle (IA) et des techniques d'apprentissage automatique offre de nouvelles opportunités pour améliorer les prévisions épidémiologiques.

Cette thèse vise à développer et évaluer un modèle prédictif pour les épidémies de chikungunya en utilisant des approches de régression d'ensemble. Les méthodes d'ensemble combinent plusieurs algorithmes d'apprentissage pour améliorer la précision prédictive par rapport aux modèles individuels. La recherche exploite des ensembles de données complets comprenant des variables épidémiologiques, climatiques et environnementales pour former et valider des modèles prédictifs. Les principaux algorithmes d'apprentissage automatique tels que Random Forest, Gradient Boosting et XGBoost sont mis en œuvre et comparés afin d'identifier l'approche la plus efficace.

L'étude commence par une exploration détaillée de l'épidémiologie du chikungunya, de la dynamique de transmission, des manifestations cliniques et des stratégies de contrôle actuelles. Elle passe en revue la littérature existante sur les applications de l'IA dans la prédiction des maladies, en soulignant le potentiel des techniques de régression d'ensemble pour améliorer la précision et la fiabilité des prévisions épidémiologiques.

Grâce à une méthodologie rigoureuse et à une analyse approfondie des résultats, cette thèse évalue la performance des modèles de régression d'ensemble dans la prévision des épidémies de chikungunya. Des mesures de performance telles que la sensibilité, la spécificité et la valeur prédictive sont utilisées pour évaluer l'efficacité et la robustesse des modèles. En outre, l'étude examine l'interprétation des résultats des modèles et discute des implications pour les interventions de santé publique.

Les résultats de cette recherche contribuent à faire progresser le domaine de la prédiction des maladies infectieuses en démontrant l'efficacité des techniques de régression d'ensemble pour améliorer la prédiction des épidémies de chikungunya.

\textbf{mot clés }: \textit{IA(Intelligence artificiel),Chikungunya,régression, ensemble techniques}

%\end{abstract}
