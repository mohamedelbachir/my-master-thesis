\chapter*{Remerciements}
\addcontentsline{toc}{chapter}{Remerciements}

\renewcommand{\abstractnamefont}{\normalfont\Large\bfseries}
%\renewcommand{\abstracttextfont}{\normalfont\Huge}


\hskip7mm
\begin{spacing}{1.3}

La réalisation de ce mémoire a été un long voyage, parsemé de défis et d'apprentissages, et n'aurait pas été possible sans le soutien et l'assistance de nombreuses personnes. C'est avec une profonde gratitude que je tiens à exprimer mes sincères remerciements à tous ceux qui ont contribué, de près ou de loin, à la concrétisation de ce travail.

\begin{itemize}
    \item Tout d'abord, je remercie Allah le Tout-Puissant qui m'a donné la volonté et la force nécessaires pour parfaire ce travail et le mener à terme ;
    \item Le Doyen de la Faculté des Sciences de l’Université de Ngaoundéré, le Professeur NGAMENI Emmanuel, pour son soutien et ses encouragements ;
    \item Le Chef du Département de Mathématiques et Informatique de la Faculté des Sciences de l’Université de Ngaoundéré : le Pr. Dr. Ing DAYANG Paul pour le suivi de notre formation et la supervision de ce travail ;
    \item Mon directeur de mémoire, Dr. ABBOUBAKAR Hamadjam, Chargé de Cours à l’Université de Ngaoundéré, au Département de Génie Informatique à l’Institut Universitaire de Technologie, pour m'avoir proposé ce sujet et pour ses précieux conseils ;
    \item Les enseignants du Département de Mathématiques et Informatique de la Faculté des Sciences de l’Université de Ngaoundéré, ainsi que les membres du jury, pour avoir accepté d'examiner ce travail ;
    \item Ma famille, pour leurs nombreux conseils, encouragements et soutien tout au long de mes études ;
    \item Mes camarades de promotion, pour la convivialité, le partage de connaissances et l'entraide dont nous avons bénéficié ensemble.
\end{itemize}

À tous, un grand merci.
\end{spacing}
