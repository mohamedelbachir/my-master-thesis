\chapter*{Abstract}
\addcontentsline{toc}{chapter}{Abstract}

\renewcommand{\abstractnamefont}{\normalfont\Large\bfseries}
%\renewcommand{\abstracttextfont}{\normalfont\Huge}

%\begin{abstract}
\hskip7mm
%

\begin{spacing}{1.3}
	
The chikungunya virus, transmitted mainly by the mosquitoes \textbf{Aedes aegypti} and \textbf{Aedes albopictus}, represents a growing threat to global public health due to its rapid spread and debilitating effects. Recently, epidemics have been reported not only in Central and East Africa, but also in South America and Southeast Asia. Predicting these epidemics remains a major challenge due to the complex interplay between environmental, climatic and biological factors. Traditional epidemiological surveillance approaches often prove insufficient to proactively anticipate epidemics.

This research focuses on \textbf{Chad}, \textbf{Brazil}, and \textbf{Paraguay}, three countries where chikungunya has emerged as a significant public health problem. Due to limited data availability in Chad, additional data from Brazil and Paraguay are incorporated to strengthen the analysis. The aim of this thesis is to develop and evaluate predictive models for chikungunya epidemics using advanced machine learning techniques, in particular ensemble regression.

The models chosen for this study include the \textbf{Random Forest Regressor}, the \textbf{XGBoost Regressor} optimized via \textbf{Grid Search}, as well as an ensemble model (\textbf{Voting Regressor}) combining \textbf{Linear Regression}, \textbf{Decision Tree Regressor}, and the optimized \textbf{XGBoost Regressor}. These models will be trained and validated using epidemiological and climatic data.

This thesis begins with an in-depth analysis of the epidemiology of chikungunya and its transmission dynamics, followed by a review of current applications of artificial intelligence in disease prediction. The models are then tested and evaluated on the basis of performance criteria such as sensitivity, specificity and predictive value.

The results of this research demonstrate the effectiveness of ensemble regression techniques in improving the accuracy of epidemiological forecasts, and offer new perspectives for public health intervention in affected regions.

\textbf{Keywords} : \textit{Chikungunya, Ensemble regression, Random Forest, XGBoost, Machine learning, Climatic data, Epidemiological forecast}

%\textbf{Keywords:} \textit{Machine learning, AI.}
\end{spacing}
%\end{abstract}
