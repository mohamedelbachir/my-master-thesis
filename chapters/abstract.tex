\chapter*{Abstract}
\addcontentsline{toc}{chapter}{Abstract}

\renewcommand{\abstractnamefont}{\normalfont\Large\bfseries}
%\renewcommand{\abstracttextfont}{\normalfont\Huge}

%\begin{abstract}
\hskip7mm
%

\begin{spacing}{1.3}
Chikungunya virus, transmitted by \textbf{Aedes aegypti} and \textbf{Aedes albopictus} mosquitoes, poses a significant public health threat globally due to its rapid spread and debilitating impact on affected populations. In recent years, outbreaks have afflicted populations in East and Central Africa, South America and Southeast Asia ~\cite{intro}. Predicting chikungunya epidemics is a challenge, due to the complex interaction of environmental, social and biological factors. Traditional epidemiological surveillance methods, while essential, may be insufficient to proactively predict epidemics. This research focuses specifically on \textbf{Chad}, a country where chikungunya has recently emerged as a major public health concern. With the increase in chikungunya cases in Chad, it is crucial to develop accurate predictive models to anticipate epidemics and implement effective control measures. The advent of artificial intelligence (AI) and machine learning techniques offers new opportunities to improve epidemiological forecasting.

This thesis aims to develop and evaluate a predictive model for chikungunya epidemics using ensemble regression approaches. Ensemble methods combine several learning algorithms to improve predictive accuracy over individual models. The research exploits comprehensive datasets including epidemiological, climatic and environmental variables to train and validate predictive models. Leading machine learning algorithms such as Random Forest, Gradient Boosting and XGBoost are implemented and compared to identify the most effective approach.

The study begins with a detailed exploration of chikungunya epidemiology, transmission dynamics, clinical manifestations and current control strategies. It reviews the existing literature on AI applications in disease prediction, highlighting the potential of ensemble regression techniques to improve the accuracy and reliability of epidemiological forecasts.

Through a rigorous methodology and an in-depth analysis of the results, this thesis evaluates the performance of ensemble regression models in predicting chikungunya epidemics. Performance metrics such as sensitivity, specificity and predictive value are used to assess model efficiency and robustness. In addition, the study examines the interpretability of model results and discusses implications for public health interventions.

The findings of this research contribute to advancing the field of infectious disease prediction by demonstrating the effectiveness of ensemble regression techniques for improving the prediction of chikungunya epidemics.

%\textbf{Keywords:} \textit{Machine learning, AI.}
\end{spacing}
%\end{abstract}
