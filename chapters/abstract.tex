\chapter*{Abstract}
\addcontentsline{toc}{chapter}{Abstract}

\renewcommand{\abstractnamefont}{\normalfont\Large\bfseries}
%\renewcommand{\abstracttextfont}{\normalfont\Huge}

%\begin{abstract}
\hskip7mm
%

\begin{spacing}{1.3}
	The chikungunya virus, primarily transmitted by female mosquitoes \textit{Aedes aegypti} and \textit{Aedes albopictus}, poses a growing threat to global public health due to its rapid spread and debilitating effects\footnote{weakening effects}. Recently, outbreaks have been reported not only in Central and East Africa but also in South America and Southeast Asia. Predicting these outbreaks remains a major challenge due to the complex interaction between environmental, climatic, and biological factors. Traditional epidemiological surveillance approaches often prove insufficient to proactively anticipate outbreaks. This research focuses on \textbf{Chad}, \textbf{Brazil}, and \textbf{Paraguay}, three countries where chikungunya has emerged as a significant public health issue. Due to the limited availability of data in Chad, additional data from Brazil and Paraguay are incorporated to strengthen the analysis. The goal here is to develop and evaluate predictive models for chikungunya outbreaks using advanced machine learning techniques, particularly ensemble regression. The models selected for this study include the \textbf{Random Forest Regressor} and the \textbf{XGBoost Regressor}, optimized via \textbf{Grid Search}, as well as an ensemble model (\textbf{Voting Regressor}) combining \textbf{Linear Regression}, \textbf{Random Forest Regressor}, and the optimized \textbf{XGBoost Regressor}. These models will be trained and validated using epidemiological and climate data. Among these models, our ensemble model \textbf{Voting Regressor}, which combines the predictions of the \textbf{Linear Regression}, \textbf{Random Forest Regressor}, and \textbf{XGBoost Regressor} models, showed overall superior performance compared to the individual models, with a minimal \textbf{MAE}, a relatively low \textbf{RMSE}, and very high accuracy (\textbf{65\%} for Brazil, \textbf{0.282\%} for Chad, and \textbf{59.57\%} for Paraguay). Epidemiological and statistical analyses revealed a correlation between climatic conditions and the number of chikungunya cases.
	
	\textbf{Keywords}: \textit{Chikungunya, Grid Search, Ensemble Regression, Random Forest, XGBoost, Machine Learning, Voting Regressor}
	
\end{spacing}
%\end{abstract}
