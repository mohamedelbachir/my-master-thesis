\chapter*{Conclusion Générale}
\addcontentsline{toc}{chapter}{Conclusion Générale et Perspectives}

L'objectif principal de cette étude était d'examiner l'influence des variables climatiques sur la propagation de la Chikungunya au Brésil, au Paraguay et au Tchad, en s'appuyant sur des modèles de régression avancés. Une base de données, intégrant des informations climatiques quotidiennes ainsi que des enregistrements épidémiologiques irréguliers, a été constituée pour ces régions. Les données épidémiologiques ont été obtenues dans les site du monitoring des cas d'épidomologie en temps réel (PACO,CDC), tandis que les données climatiques provenaient de sources fiables telles que le site WeatherAndClimate.

La méthodologie adoptée, qui combine des modèle de\textit{machine learning} telles que le \textit{Voting Regressor}, a permis de dégager plusieurs résultats significatifs. Notamment, il a été observé qu'une relation existe entre les variations des paramètres climatiques, comme la température et l'humidité, et l'évolution des cas de Chikungunya, notre étude démontre également que le modèles d'ensemble, et en particulier le \textit{Voting Regressor}, peuvent fournir des prédictions relativement précises pour les cas de Chikungunya en s'appuyant sur des données climatiques et épidémiologiques. Cependant, des améliorations supplémentaires, telles que l'intégration de méthodes d'apprentissage plus avancées ou l'inclusion de variables explicatives supplémentaires, pourraient encore renforcer la précision et la fiabilité des prédictions.

Cependant, bien que cette étude se soit concentrée sur les variables climatiques comme principaux facteurs explicatifs, il est évident que d'autres éléments contribuent également à la propagation du Chikungunya. Parmi ces facteurs, on peut citer l'accès aux soins de santé, les conditions socio-économiques, les pratiques environnementales, et les dynamiques locales des vecteurs de la maladie tels que les moustiques \textit{Aedes}. Pour de futures recherches, il serait pertinent d'inclure ces variables supplémentaires pour mieux comprendre la complexité de la propagation du Chikungunya.

Vu ces resultats , plusieurs recommandations sont formulées pour les chercheurs, les décideurs et les communautés locales.

\begin{itemize}
	\item Pour les \textbf{chercheurs} et les \textbf{décideurs}, il est suggéré de renforcer la surveillance des cas de Chikungunya en mettant en place des systèmes d'alerte précoce et en améliorant la collecte des données épidémiologiques et climatiques. Il est également recommandé de développer des stratégies adaptatives utilisant les nouvelles approches de l'intelligence artificielle pour mieux prédire et gérer les épidémies de Chikungunya.
	\item Pour les \textbf{communautés}, il est conseillé de maintenir un environnement propre, de promouvoir l'utilisation régulière de moustiquaires imprégnées d'insecticide, et de sensibiliser davantage aux modes de transmission et aux méthodes de prévention du Chikungunya.
\end{itemize}

Cette étude contribue à une meilleure compréhension de l'impact des variables climatiques sur la propagation du Chikungunya et propose des pistes pour améliorer la gestion et la prévention de cette maladie dans les régions étudiées.






