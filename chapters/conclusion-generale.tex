\chapter*{Conclusion Générale et perspectives}
\addcontentsline{toc}{chapter}{Conclusion Générale et Perspectives}

L'objectif principal de cette étude était de développer un modèle prédictif du chikungunya, en utilisant des approches de régression d'ensemble issues de l'intelligence artificielle en examinant l'influence des variables climatiques sur la propagation du Chikungunya au Brésil, au Paraguay et au Tchad, en s'appuyant sur des modèles de \textbf{régression avancés}. 

Les données épidémiologiques ont été obtenues sur le site du monitoring des cas d'épidomologie en temps réel PAHO (pour le Paraguay) ,dans le rapport d'OMS(pour le cas du Tchad) et dans le site mendeley pour celui du brésil tandis que les données climatiques provenaient de sources fiables telles que le site WeatherAndClimate.  Les modèles choisis pour cette étude incluaient le \textbf{Random Forest Regressor} et le \textbf{XGBoost Regressor} optimisé via \textbf{Grid Search}, ainsi qu'un modèle d'ensemble (\textbf{Voting Regressor}) combinant \textbf{Linear Regression}, \textbf{Random Forest Regressor} et le \textbf{XGBoost Regressor} optimisé. Parmi ces modèles, notre modèle d'ensemble \textbf{Voting Regressor}, qui combine les prédictions des modèles \textbf{Linear Regression}, \textbf{Random Forest Regressor} et \textbf{XGBoost Regressor}, a affiché des performances globalement supérieures aux autres modèles individuels, avec un \textbf{MAE} minimal, un \textbf{RMSE} relativement bas et une très bonne précision (\textbf{91,08\%} pour le Brésil, \textbf{34,34\%} pour le Tchad et \textbf{59,57\%} pour le Paraguay).

Les limites de l’étude ont montré que les variables climatiques ne suffisent pas à elles seules à expliquer les variations des cas de chikungunya dans ces pays. Il est proposé, pour les futures
recherches, d’intégrer d’autres facteurs environnementaux et de développer un modèle hybride combinant des algorithmes d’apprentissage automatique pour affiner les prévisions. Les recommandations incluent l’amélioration de la collecte des données, l’adoption de stratégies adaptées aux variations climatiques, et la sensibilisation des communautés sur l’hygiène et la prévention du chikungunya.







