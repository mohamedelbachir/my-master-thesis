\chapter*{Introduction Générale}
\addcontentsline{toc}{chapter}{General Introduction} 

Le chikungunya est une maladie virale transmise par la piqûre d'un moustique infecté et endémique en Afrique de l'Est et dans certaines régions d'Asie. Les symptômes du chikungunya sont la fièvre, des douleurs articulaires, des maux de tête, des douleurs musculaires, un gonflement des articulations ou une éruption cutanée. Les décès dus au chikungunya sont rares. Le virus peut entraîner de graves problèmes chez certaines personnes, en particulier les personnes âgées et les personnes souffrant d'autres maladies chroniques. Une détection précoce du chikungunya peut empêcher la maladie de se propager. Notre projet vise à prédire si une personne est atteinte ou non du chikungunya et à prendre des mesures de précaution en fonction du résultat. Il n'existe pas de traitement antiviral spécifique pour le chikungunya. La prise en charge clinique se concentre sur le soulagement des symptômes, comme le soulagement des douleurs articulaires à l'aide d'antipyrétiques, d'analgésiques appropriés, d'une consommation suffisante de liquides et d'une relaxation générale~\cite{9985708}.
Les progrès récents dans le domaine de l'intelligence artificielle (IA) ont permis d'améliorer considérablement les prédictions grâce à des algorithmes qui se sont révélés capables de saisir les relations non linéaires entre les données d'entrée et de sortie. Ces algorithmes sont généralement connus sous le nom d'algorithmes d'apprentissage automatique qui sont basés sur des données historiques et actuelles.

L'objectif de ce mémoire de master est de développer un modèle de prédiction du chikungunya. Pour ce faire, nous avons opté pour une méthode d'apprentissage automatique issue de l'intelligence artificielle et basée sur l'approche de régression d'ensemble.

Dans ce travail, nous nous intéressons à répondre à la problématique : Comment utiliser l'apprentissage automatique par l'approche ensembliste pour prédire et analyser le chikungunya ?

Les objectifs spécifiques sont donc les suivants.
\begin{itemize}
	
	\item Comprendre les concepts liés à la régression d'ensemble et au chikungunya ;
	\item Prédire le chikungunya au Tchad ;	
	\item Analyser la situation du chikungunya au Tchad.
\end{itemize}
	
Le reste du manuscrit est organisé comme suit. Le premier chapitre est consacré à l'épidémiologie du chikungunya et à un focus sur le cas du Tchad. Le deuxième chapitre donne une introduction à l'apprentissage automatique pour les maladies. Dans le troisième chapitre, nous présentons la conception générale et détaillée du modèle proposé. Dans le quatrième chapitre, nous illustrons l'efficacité de notre modèle par des résultats de simulation. Le travail se termine par une conclusion générale et quelques perspectives.

