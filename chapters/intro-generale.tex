\chapter*{Introduction Générale}
\addcontentsline{toc}{chapter}{General Introduction} 

Le chikungunya est une maladie virale transmise par la piqûre de moustiques infectés, principalement des espèces Aedes aegypti et Aedes albopictus. Endémique dans certaines régions d'Afrique de l'Est, d'Asie et plus récemment d'Amérique du Sud, le chikungunya se manifeste par des symptômes tels que fièvre, douleurs articulaires, maux de tête, douleurs musculaires, gonflements articulaires, et éruptions cutanées. Bien que les décès dus au chikungunya soient rares, le virus peut provoquer de graves complications, en particulier chez les personnes âgées ou celles souffrant de maladies chroniques. Une détection précoce est cruciale pour prévenir la propagation de la maladie.

L'objectif de ce mémoire est de développer un modèle prédictif du chikungunya, en utilisant des approches de régression d'ensemble issues de l'intelligence artificielle. Compte tenu du manque de données suffisantes pour le Tchad, nous avons étendu notre étude aux données du Brésil et du Paraguay, où des cas de chikungunya ont également été signalés. Cette approche nous permet de tirer parti d'un ensemble de données plus vaste et diversifié pour améliorer la précision de nos prévisions.

Nous nous posons donc la question suivante : comment utiliser les techniques d'apprentissage automatique, en particulier l'approche ensembliste, pour prédire et analyser les épidémies de chikungunya ?

Les objectifs spécifiques de ce travail sont les suivants :
\begin{itemize}
	\item Comprendre les concepts liés à la régression d'ensemble et au chikungunya ;
	\item Prédire les épidémies de chikungunya en se basant sur les données climatiques et les cas rapportés au Tchad, au Brésil, et au Paraguay ;
	\item Analyser la dynamique du chikungunya dans ces régions et proposer des stratégies d'intervention.
\end{itemize}

Le reste du document est structuré comme suit : le premier chapitre est dédié à l'épidémiologie du chikungunya avec un accent particulier sur les cas au Tchad, au Brésil et au Paraguay. Le deuxième chapitre introduit l'apprentissage automatique dans le contexte des maladies infectieuses. Le troisième chapitre présente la conception générale et détaillée du modèle proposé. Dans le quatrième chapitre, nous illustrons l'efficacité de notre modèle à travers des résultats de simulation. Enfin, le mémoire se conclut par une synthèse des résultats et des perspectives pour des recherches futures.

