\chapter{Revue de la Littérature et Concepts de Base}

\section{Introduction}
\subsection{Objectifs du Chapitre}
\subsection{Méthodologie de la Revue}

\section{Machine Learning}
\subsection{Définition et Concepts Fondamentaux}
\subsection{Techniques de Machine Learning}
\subsubsection{Régression Linéaire}
\subsubsection{Régression Logistique}
\subsubsection{Arbres de Décision}
\subsubsection{Forêts Aléatoires}
\subsection{Applications du Machine Learning à l'Épidémiologie}
\subsection{Avantages et Limites du Machine Learning}

\section{Deep Learning}
\subsection{Définition et Concepts Fondamentaux}
\subsection{Réseaux de Neurones Artificiels}
\subsubsection{Perceptron Multicouche (MLP)}
\subsubsection{Réseaux de Neurones Convolutifs (CNN)}
\subsubsection{Réseaux de Neurones Récurrents (RNN)}
\subsection{Applications du Deep Learning à l'Épidémiologie}
\subsection{Avantages et Limites du Deep Learning}

\section{Ensemble Regression Learning}
\subsection{Définition et Concepts Fondamentaux}
\subsection{Techniques d'Ensemble Learning}
\subsubsection{Bagging}
\subsubsection{Boosting}
\subsubsection{Stacking}
\subsection{Applications de l'Ensemble Regression Learning}
\subsection{Avantages et Limites de l'Ensemble Regression Learning}

\section{Comparaison des Approches}
\subsection{Machine Learning vs Deep Learning}
\subsection{Machine Learning et Deep Learning vs Ensemble Learning}

\section{Études de Cas et Travaux Antérieurs}
\subsection{Études sur le Chikungunya}
\subsection{Applications Pratiques des Techniques d'Ensemble}

\section{Conclusion}
\subsection{Synthèse des Connaissances Actuelles}
\subsection{Orientations Futures et Opportunités de Recherche}