\documentclass[a4paper,12pt]{report}

%====================== PACKAGES ======================
\usepackage{multirow}
\usepackage[french]{babel}
\usepackage[utf8x]{inputenc}
%pour gérer les positionnement d'images
\usepackage{float}
\usepackage{amsmath}
\usepackage{graphicx}
\usepackage[colorinlistoftodos]{todonotes}
\usepackage{url}
%pour les informations sur un document compilé en PDF et les liens externes / internes
\usepackage{hyperref}
%pour la mise en page des tableaux
\usepackage{array}
\usepackage{tabularx}
\usepackage{enumitem}
%pour utiliser \floatbarrier
%\usepackage{placeins}
%\usepackage{floatrow}
%espacement entre les lignes
\usepackage{setspace}
%modifier la mise en page de l'abstract
\usepackage{abstract}
%police et mise en page (marges) du document
%\usepackage[T1]{fontenc}
\usepackage[top=2cm, bottom=2cm, left=2cm, right=2cm]{geometry}
%Pour les galerie d'images
\usepackage{subfig}

\usepackage{indentfirst}
\usepackage{algorithm}
\usepackage{algpseudocode}
\usepackage{longtable}
\usepackage{babel}


%pour l'ajout de la table des matières dans la table des matières ndam njoya
\usepackage{tocbibind} 


%====================== INFORMATION ET REGLES ======================

%rajouter les numérotation pour les \paragraphe et \subparagraphe
\setcounter{secnumdepth}{4}
\setcounter{tocdepth}{4}
\hypersetup{                            % Information sur le document
pdfauthor = {Mohamed el bachir},          % Auteurs
pdftitle = {Memoire - Deep Analysis and prediction of chikungunya using ensemble regression tools},           % Titre du document
pdfsubject = {Mémoire},       % Sujet
pdfstartview={FitW},                    % ajuste la page à la largueur de l'écran
hidelinks}                              % désactive les cadres autour des liens
%pdfcreator = {MikTeX},% Logiciel qui a créé le document
%pdfproducer = {}} % Société qui produit le logiciel

%======================== DEBUT DU DOCUMENT ========================

\begin{document}

% régler l'espacement entre les lignes
\newcommand{\HRule}{\rule{\linewidth}{0.5mm}}

% page de garde
\begin{titlepage}
\begin{center}


\begin{minipage}{0.38\textwidth}
\begin{centering} 
\bf{UNIVERSITÉ DE NGAOUNDÉRÉ}\\[0.2cm]
\bf{\it{FACULTÉ DES SCIENCES}}\\[0.2cm]
\bf{\it{DÉPARTEMENT DE MATHEMATIQUES ET INFORMATIQUE}}\\[0.2cm]
\end{centering}
\end{minipage}
\begin{minipage}{0.20\textwidth}
	\centering
	\includegraphics[width=0.30\textwidth]{./LogoUnivNg}\\
	\includegraphics[width=0.30\textwidth]{./LogoUnivNgFS}
	
\end{minipage}
\begin{minipage}{0.38\textwidth}
\begin{centering} 
\bf{THE UNIVERSITY OF NGAOUNDERE}\\[0.2cm]
\bf{\it{FACULTY OF SCIENCE}}\\[0.2cm]
\bf{\it{{DEPARTMENT OF MATHEMATICS AND COMPUTER SCIENCE}}}\\[0.2cm]
\end{centering}
\end{minipage}\\[1.5cm]

\textsc{\Large \textbf{ÉCOLE DOCTORALE : SCIENCES, TECHNOLOGIE ET INGÉNIERIE (STI)}}\\[0.1cm]
\textsc{\Large \textbf{UFD : MATHÉMATIQUES, INFORMATIQUE, INGÉNIERIE ET APPLICATION (M2IAP)}}\\[0.8cm]
%\begin{tabularx}{\textwidth}{XcX}
%	\raggedright\bfseries{UNIVERSITÉ DE NGAOUNDÉRÉ} & & \raggedleft\bfseries{THE UNIVERSITY OF NGAOUNDERE} \\[0.2cm]
%	\raggedright\bfseries\itshape{FACULTÉ DES SCIENCES} & & \raggedleft\bfseries\itshape{FACULTY OF SCIENCE} \\[0.2cm]
%	\raggedright\itshape{DÉPARTEMENT DE MATHEMATIQUES ET INFORMATIQUE} & & \raggedleft\itshape{DEPARTMENT OF MATHEMATICS AND COMPUTER SCIENCE} \\[0.5cm]
%	& \begin{minipage}{1cm}\centering\includegraphics[width=0.10\textwidth]{./LogoUnivNg}\\\includegraphics[width=0.10\textwidth]{./LogoUnivNgFS}\end{minipage} & \\[0.5cm]
%\end{tabularx}



% Upper part of the page. The '~' is needed because only works if a paragraph has started.
%\includegraphics[width=0.10\textwidth]{./LogoUnivNg}~\\[0.8cm]
%\includegraphics[width=0.10\textwidth]{./LogoUnivNgFS}~\\[0.8cm]
%\includegraphics[width=0.35\textwidth]{./LogoUnivNgFS}~\\[1cm]


%\sf{\large \textbf{Unité de Formation Doctorale Mathématiques, Informatique et Applications}}\\[0.2cm]
%\sf{\large   \textbf{\textit{Doctoral Training Unit in Mathematics, Computer Science and Applications}}}\\[0.8cm]

%\textsc{\Large }\\[0.5cm]

% Title
\HRule \\[0.2cm]

{\huge \bfseries Deep Analysis and prediction of chikungunya using ensemble regression approach  \\[0.2cm] }

\HRule \\[1cm]

\sf{\Large \textbf{Mémoire en vue d'obtention du Master II}}\\[0.4cm]
\sf{\large \textbf{Filière: Informatique}}\\[0.4cm]
\sf{\Large \textbf{Spécialité: Systèmes et  Logiciels en Environnements distribués}}\\[0.8cm]

\sf{\Large \textbf{Par}}\\[0.8cm]

\sf{\Large \textbf{MOHAMED EL BACHIR BOUBA NGANADAKOUA}}\\[0.4cm]
\sf{\large ( Master I Systèmes et  Logiciels en Environnements distribués ) }\\[0.4cm]
\sf{\Large \textbf{Matricule: 19A666FS}}\\[0.8cm]

\sf{\Large \textbf{Sous la direction de  :}}\\[0.8cm]

\begin{tabularx}{\textwidth}{lXr}
	\sc{Pr. Dr. Ing Dayang Paul} & & \sc{Dr. ABBOUBAKAR Hamadjam} \\[0.2cm]
	Maître de Conférence & & Chargé de Cours \\[0.2cm]
	Faculté des Sciences & & Institut Universitaire de Technologie \\[0.2cm]
	Université de Ngaounderé & & Université de Ngaounderé \\[0.2cm]
\end{tabularx}


% Author and supervisor
%\begin{minipage}{0.1\textwidth}
%\begin{flushleft} \large
%\emph{Auteur:}\\
%Premier \textsc{Auteur}\\
%Deuxième \textsc{Auteur}\\
%Troisième \textsc{Auteur}\\
%Quatrième \textsc{Auteur}
%\end{flushleft}
%\end{minipage}
%\begin{minipage}{0.1\textwidth}
%\begin{flushright} \large
%\emph{Client:} \\
%Prénom \textsc{Nom}\\
%\emph{Référent:} \\
%Prénom \textsc{Nom}
%\end{flushright}
%\end{minipage}

\vfill

% Bottom of the page
%{\large \today}
{\large \bf{Année académique 2023-2024}}

\end{center}
\end{titlepage}

% page blanche
%\newpage % ndam njoya
%~
% ne pas numéroter cette page
%\thispagestyle{empty} % ndam njoya

\pagenumbering{roman} \setcounter{page}{1}
%\newpage % ndam njoya

\begin{spacing}{1.5}

\chapter*{Dédicace}
\addcontentsline{toc}{chapter}{Dédicace}

\renewcommand{\abstractnamefont}{\normalfont\Large\bfseries}
%\renewcommand{\abstracttextfont}{\normalfont\Huge}


\hskip7mm
\begin{spacing}{1.3}
	\begin{flushright}
	\itshape
à mon père BOUBA Nana Dekwa,\\
à ma mère HABIBA Kouyessi,\\
à mes sœurs.\\
	\end{flushright}
\end{spacing}


\chapter*{Remerciements}
\addcontentsline{toc}{chapter}{Remerciements}

\renewcommand{\abstractnamefont}{\normalfont\Large\bfseries}
%\renewcommand{\abstracttextfont}{\normalfont\Huge}


\hskip7mm
\begin{spacing}{1.3}
Nous exprimons notre profonde gratitude à Allah le Tout-Puissant, source d’inspiration et de force, qui a rendu possible l’accomplissement de ce travail. Nos remerciements les plus sincères vont à l’endroit de : 
\begin{itemize}
	\item Monsieur le Doyen de la Faculté des Sciences de l’Université de Ngaoundéré, pour la qualité de l’enseignement dispensé ;
    \item Pr. Dr. Ing DAYANG PAUL, Chef du Département de Mathématiques et Informatique de la Faculté des Sciences de l’Université de Ngaoundéré ;
    \item Mon directeur de mémoire, Dr. ABBOUBAKAR Hamadjam, Chargé de Cours à l’Université de Ngaoundéré, au Département de Génie Informatique à l’Institut Universitaire de Technologie, pour m'avoir proposé ce sujet de recherche et pour ses précieux conseils ;
    \item Mon Co-directeur Dr. ZONGO MEYO Epse NDO pour son soutien; 
    \item Les enseignants du Département de Mathématiques et Informatique de la Faculté des Sciences de l’Université de Ngaoundéré, ainsi que les membres du jury, pour avoir accepté d'examiner ce travail ;
    \item À mon oncle ABDEL FADEL LIMAKAMATA pour son soutien et son encouragement ;
    \item À ma famille, pour leurs précieux conseils, leur soutien indéfectible et leurs encouragements constants tout au long de mes études ;
    \item Mes camarades de promotion, pour la convivialité, le partage de connaissances et l'entraide dont nous avons bénéficié ensemble.
\end{itemize}

À tous, un grand merci.
\end{spacing}


\addtocontents{toc}{\protect\setcounter{tocdepth}{-1}}
\tableofcontents
\addtocontents{toc}{\protect\setcounter{tocdepth}{2}}
%\thispagestyle{empty}
%\setcounter{page}{0}
% ne pas numéroter le sommaire

\renewcommand\listfigurename{Liste de figures}
\listoffigures % ndam njoya

\renewcommand\listfigurename{List of tables}
\listoftables % ndam njoya

\newpage
\chapter*{Notation et abr\'{e}viations}
\addcontentsline{toc}{chapter}{Notation et abr\'{e}viations}
\renewcommand{\abstractnamefont}{\normalfont\Large\bfseries}
%\renewcommand{\abstracttextfont}{\normalfont\Huge}
%\chapter*{}
%\addcontentsline{toc}{chapter}{Notation et abbr�viations}

\hskip7mm
\begin{spacing}{1.3}
	
	\begin{tabular}{ll}
		
		\textbf{\textit{Ae}} & Aedes \\
		\textbf{CDC} & Centers for Disease Control and Prevention \\
		\textbf{CHIKV} & Chikungunya virus \\
		\textbf{CHIKVD} & Chikungunya Virus Diseases \\
		\textbf{CNN} & Convolution neural network \\
		\textbf{NLP} & Natural Language Processing \\
		\textbf{ML} & Machine learning \\
		\textbf{\textit{MoE}} & Mixture of Experts \\
		\textbf{OMS} & Organisation Mondiale de la Sante \\
		\textbf{PAHO} & Pan American Health Organization \\
		\textbf{ReLu} & Rectified linear unit \\
		\textbf{RNA} & R\'{e}seau de Neurone Artificiel \\
		\textbf{RMSE} & Root Squared Mean Error \\
		\textbf{RNN} & Recurrent neural network \\
		\textbf{WHO} & World Health Organization \\
		%\textbf{BCO} & Bee Colony Optimization : Algorithmes de colonie d�abeilles\\
		
		
	\end{tabular}
\end{spacing}

\chapter*{Résumé}
\addcontentsline{toc}{chapter}{Résumé}

\renewcommand{\abstractnamefont}{\normalfont\Large\bfseries}
%\renewcommand{\abstracttextfont}{\normalfont\Huge}

%\begin{abstract}
\hskip7mm

\begin{spacing}{1.3}

\end{spacing}

Le virus du chikungunya, transmis principalement par les moustiques \textbf{Aedes aegypti} et \textbf{Aedes albopictus}, représente une menace croissante pour la santé publique mondiale en raison de sa propagation rapide et de ses effets débilitants. Récemment, des épidémies ont été signalées non seulement en Afrique centrale et orientale, mais aussi en Amérique du Sud et en Asie du Sud-Est. Prédire ces épidémies reste un défi majeur en raison de l'interaction complexe entre les facteurs environnementaux, climatiques et biologiques. Les approches traditionnelles de surveillance épidémiologique se révèlent souvent insuffisantes pour anticiper les épidémies de manière proactive.

Cette recherche se concentre sur le \textbf{Tchad}, le \textbf{Brésil}, et le \textbf{Paraguay}, trois pays où le chikungunya a émergé comme un problème significatif de santé publique. En raison de la disponibilité limitée de données au Tchad, des données supplémentaires provenant du Brésil et du Paraguay sont intégrées pour renforcer l'analyse. Le but de cette thèse est de développer et d'évaluer des modèles prédictifs pour les épidémies de chikungunya en utilisant des techniques avancées d'apprentissage automatique, en particulier la régression d'ensemble.

Les modèles choisis pour cette étude incluent le \textbf{Random Forest Regressor}, le \textbf{XGBoost Regressor} optimisé via \textbf{Grid Search}, ainsi qu'un modèle d'ensemble (\textbf{Voting Regressor}) combinant \textbf{Linear Regression}, \textbf{Decision Tree Regressor}, et le \textbf{XGBoost Regressor} optimisé. Ces modèles seront formés et validés à partir de données épidémiologiques et climatiques.

Cette thèse commence par une analyse approfondie de l'épidémiologie du chikungunya et de ses dynamiques de transmission, suivie d'une revue des applications actuelles de l'intelligence artificielle dans la prédiction des maladies. Les modèles sont ensuite testés et évalués sur la base de critères de performance tels que la sensibilité, la spécificité, et la valeur prédictive.

Les résultats de cette recherche montrent l'efficacité des techniques de régression d'ensemble pour améliorer la précision des prévisions épidémiologiques, et offrent des perspectives nouvelles pour l'intervention en santé publique dans les régions affectées.

\textbf{Mots-clés} : \textit{Chikungunya, Régression d'ensemble, Random Forest, XGBoost, Apprentissage Automatique, Données Climatiques, Prévision Épidémiologique}

%\end{abstract}

\chapter*{Abstract}
\addcontentsline{toc}{chapter}{Abstract}

\renewcommand{\abstractnamefont}{\normalfont\Large\bfseries}
%\renewcommand{\abstracttextfont}{\normalfont\Huge}

%\begin{abstract}
\hskip7mm
%

\begin{spacing}{1.3}
	
The chikungunya virus, transmitted mainly by the mosquitoes \textbf{Aedes aegypti} and \textbf{Aedes albopictus}, represents a growing threat to global public health due to its rapid spread and debilitating effects. Recently, epidemics have been reported not only in Central and East Africa, but also in South America and Southeast Asia. Predicting these epidemics remains a major challenge due to the complex interplay between environmental, climatic and biological factors. Traditional epidemiological surveillance approaches often prove insufficient to proactively anticipate epidemics.

This research focuses on \textbf{Chad}, \textbf{Brazil}, and \textbf{Paraguay}, three countries where chikungunya has emerged as a significant public health problem. Due to limited data availability in Chad, additional data from Brazil and Paraguay are incorporated to strengthen the analysis. The aim of this thesis is to develop and evaluate predictive models for chikungunya epidemics using advanced machine learning techniques, in particular ensemble regression.

The models chosen for this study include the \textbf{Random Forest Regressor}, the \textbf{XGBoost Regressor} optimized via \textbf{Grid Search}, as well as an ensemble model (\textbf{Voting Regressor}) combining \textbf{Linear Regression}, \textbf{Decision Tree Regressor}, and the optimized \textbf{XGBoost Regressor}. These models will be trained and validated using epidemiological and climatic data.

This thesis begins with an in-depth analysis of the epidemiology of chikungunya and its transmission dynamics, followed by a review of current applications of artificial intelligence in disease prediction. The models are then tested and evaluated on the basis of performance criteria such as sensitivity, specificity and predictive value.

The results of this research demonstrate the effectiveness of ensemble regression techniques in improving the accuracy of epidemiological forecasts, and offer new perspectives for public health intervention in affected regions.

\textbf{Keywords} : \textit{Chikungunya, Ensemble regression, Random Forest, XGBoost, Machine learning, Climatic data, Epidemiological forecast}

%\textbf{Keywords:} \textit{Machine learning, AI.}
\end{spacing}
%\end{abstract}


\newpage

% espacement entre les lignes d'un tableau
\renewcommand{\arraystretch}{1.5}

%====================== INCLUSION DES PARTIES ======================

\thispagestyle{empty}
% recommencer la numérotation des pages à "1"
\pagenumbering{arabic} % ndam njoya 
\setcounter{page}{1}
\newpage

\chapter*{Introduction Générale}
\addcontentsline{toc}{chapter}{Introduction Générale} 

Le chikungunya est une maladie virale transmise par la piqûre de moustiques infectés, principalement des espèces Aedes aegypti et Aedes albopictus.Le 2 août 2024, le Centre européen de contrôle et de prévention des maladies (ECDC) a indiqué qu'environ \textbf{350 000} cas de maladie à virus chikungunya (CHIKVD) et plus de \textbf{140 décès} ont été signalés dans le monde en 2024, Ces cas proviennent de 21 pays d'Amérique, d'Asie, d'Afrique et d'Europe~\cite{chikvecdc}.le chikungunya se manifeste par des symptômes tels que fièvre, douleurs articulaires, maux de tête, douleurs musculaires, gonflements articulaires, et éruptions cutanées. Bien que les décès dus au chikungunya soient rares, le virus peut provoquer de graves complications, en particulier chez les personnes âgées ou celles souffrant de maladies chroniques. Une détection précoce est cruciale pour prévenir la propagation de la maladie.

L'objectif de ce mémoire est de développer un modèle prédictif du chikungunya, en utilisant des approches de régression d'ensemble issues de l'intelligence artificielle. Compte tenu du manque de données suffisantes pour le Tchad, nous avons étendu notre étude aux données du Brésil et du Paraguay, où des cas de chikungunya ont également été signalés. Cette approche nous permet de tirer parti d'un ensemble de données plus vaste et diversifié pour améliorer la précision de nos prévisions.

Nous nous posons donc la question suivante : comment utiliser les techniques d'apprentissage automatique, en particulier l'approche ensembliste, pour prédire et analyser les épidémies de chikungunya ?

Les objectifs spécifiques de ce travail sont les suivants :
\begin{itemize}
	\item Comprendre les concepts liés à la régression d'ensemble et au chikungunya ;
	\item Prédire les épidémies de chikungunya en se basant sur les données climatiques et les cas rapportés au Tchad, au Brésil, et au Paraguay ;
	\item Analyser la dynamique du chikungunya dans ces régions et proposer des stratégies d'intervention.
\end{itemize}

Le reste du document est structuré comme suit : le premier chapitre est dédié à l'épidémiologie du chikungunya avec un accent particulier sur les cas au Tchad, au Brésil et au Paraguay. Le deuxième chapitre introduit l'apprentissage automatique dans le contexte des maladies infectieuses. Le troisième chapitre présente la conception générale et détaillée du modèle proposé. Dans le quatrième chapitre, nous illustrons l'efficacité de notre modèle à travers des résultats de simulation. Enfin, le mémoire se conclut par une synthèse des résultats et des perspectives pour des recherches futures.



\chapter{Épidémiologie de la Chikungunya}


\section*{Introduction}

 
\section{Origine de la Chikungunya}
\textbf{Chikungunya fever} (CHIKF) is a viral disease that was first described in \textbf{1952} during an outbreak in southern Tanzania. The name comes from a word in the \textit{Makonde} language, spoken in southeast Tanzania and northern Mozambique, that means "\textit{to become contorted}" or "\textit{that which bends up}". The virus was first isolated in Thailand in 1958.\cite{origin}

\section{Agent Pathogène}

Blajxckc

\subsection{Le virus Chikungunya}

\begin{figure}[!h]
	\begin{center}
		%taille de l'image en largeur
		%remplacer "width" par "height" pour régler la hauteur
		\includegraphics[width=10cm]{images/moustique}
	\end{center}
	%légende de l'image
	\caption{Aedes aegypti mosquito full of blood}
	\label{fig:aedes}
\end{figure}

\begin{figure}[!h]
	\begin{center}
		%taille de l'image en largeur
		%remplacer "width" par "height" pour régler la hauteur
		\includegraphics[width=10cm]{images/CHIK_17550_TEM}
	\end{center}
	%légende de l'image
	\caption{Electron microscopic image of chikungunya virus}
	\label{fig:chikv}
\end{figure}

\begin{figure}[!h]
	\begin{center}
		%taille de l'image en largeur
		%remplacer "width" par "height" pour régler la hauteur
		\includegraphics[width=10cm]{images/chadmap}
	\end{center}
	%légende de l'image
	\caption{Tools to Develop our Model}
	\label{Tools to Develop our Model}
\end{figure}

\begin{figure}[!h]
	\begin{center}
		%taille de l'image en largeur
		%remplacer "width" par "height" pour régler la hauteur
		\includegraphics[width=10cm]{images/statsCaseChad}
	\end{center}
	%légende de l'image
	\caption{Tools to Develop our Model}
	\label{Tools to Develop our Model}
\end{figure}

\begin{figure}[!h]
	\begin{center}
		%taille de l'image en largeur
		%remplacer "width" par "height" pour régler la hauteur
		\includegraphics[width=10cm]{images/trans}
	\end{center}
	%légende de l'image
	\caption{Tools to Develop Model}
	\label{Tools to Develop our Model}
\end{figure}

\begin{figure}[!h]
	\begin{center}
		%taille de l'image en largeur
		%remplacer "width" par "height" pour régler la hauteur
		\includegraphics[width=10cm]{images/zjv9990995820001}
	\end{center}
	%légende de l'image
	\caption{Tools to Develop our Model}
	\label{Tools to Develop our Model}
\end{figure}
\section{Mode de Transmission}

Bla

\subsection{Vecteurs : Les moustiques Aedes}

Bla

\subsection{La transmission}
Le \textbf{CHIKV} se transmet selon deux cycles différents :
\begin{itemize}
	\item \textbf{Cycle urbain} : transmission de l'homme au moustique.
	\item \textbf{Cycle sylvatique} : transmission de l'animal au moustique, puis à l'homme \cite{ganesan2017chikungunya}.
\end{itemize}
Le cycle sylvatique est la principale forme de transmission en Afrique \cite{ganesan2017chikungunya}. Ailleurs, dans les zones plus densément peuplées, le CHIKV se maintient principalement dans un cycle urbain, dans lequel les humains sont les principaux hôtes et les moustiques du genre \textit{Aedes} les vecteurs \cite{ganesan2017chikungunya} (voir figure\ref).
bien que \textit{Ae. aegypti} continue d'être un vecteur viral important, comme on l'a vu lors de la flambée épidémique dans les Caraïbes en 2013 \cite{ganesan2017chikungunya}.

La transmission verticale de la mère à l'enfant a été postulée pour expliquer les incidences postérieures à 2005 \cite{ganesan2017chikungunya}, étant particulièrement délétère lorsque :
\begin{itemize}
	\item la mère est infectée jusqu'à quatre jours après l'accouchement \cite{ganesan2017chikungunya},
\end{itemize}
bien que cette hypothèse ait été contestée \cite{ganesan2017chikungunya}.

\section{Symptômes et Diagnostic}
La reconnaissance des symptômes et l'établissement d'un diagnostic précis sont essentiels pour la gestion efficace des cas de chikungunya. Cette section examine les manifestations cliniques typiques de l'infection par le virus chikungunya ainsi que les méthodes diagnostiques utilisées pour identifier la maladie.
\subsection{Symptômes}
Chez les patients symptomatiques, la maladie à \textbf{CHIKV} se déclare généralement 4 à 8 jours (entre 2 et 12 jours) après la piqûre d'un moustique infecté. Elle se caractérise par une brusque poussée de fièvre, souvent accompagnée de fortes douleurs articulaires. Les douleurs articulaires sont souvent invalidantes et durent généralement quelques jours, mais peuvent être prolongées et durer des semaines, des mois, voire des années. D'autres signes et symptômes courants sont le gonflement des articulations, les douleurs musculaires, les maux de tête, les nausées, la fatigue et les éruptions cutanées. Comme ces symptômes se confondent avec ceux d'autres infections, notamment celles dues aux virus de la dengue et du Zika, les cas peuvent être mal diagnostiqués. En l'absence de douleurs articulaires importantes, les symptômes des personnes infectées sont généralement légers et l'infection peut passer inaperçue.

La plupart des patients se rétablissent complètement de l'infection ; toutefois, des cas occasionnels de complications oculaires, cardiaques et neurologiques ont été signalés dans le cadre d'infections par le \textbf{CHIKV}. Les patients situés aux extrémités du spectre d'âge sont plus exposés à une maladie grave. Les nouveau-nés infectés pendant l'accouchement et les personnes âgées souffrant de pathologies sous-jacentes peuvent devenir gravement malades et l'infection par le \textbf{CHIKV} peut augmenter le risque de décès~\cite{who2}.

Une fois qu'une personne est guérie, les données disponibles suggèrent qu'elle est probablement immunisée contre les infections futures~\cite{auerswald2018broad}.

\subsection{Méthodes de diagnostic}
ccxc
Bla

\section{Méthodes de Contrôle et Traitement}

La gestion efficace de l'épidémie de chikungunya repose sur une combinaison de stratégies de contrôle des vecteurs et d'interventions médicales. Cette section explore les diverses approches utilisées pour prévenir la transmission du virus et traiter les symptômes chez les patients infectés.

\subsection{Méthodes de contrôle}

Bla

\subsection{Options de traitement}

Bla

\section{Cas du Tchad}

L'analyse spécifique des cas de chikungunya au Tchad permet de comprendre l'impact de cette maladie dans un contexte régional spécifique. Cette section examine les caractéristiques épidémiologiques, les stratégies de contrôle et les défis rencontrés dans la gestion de l'infection par le virus chikungunya dans ce pays d'Afrique centrale.
\subsection{Facteurs climatiques influençant la propagation}

Bla

\subsection{Études de cas et données climatiques}


\chapter{Revue de la litterature et concepts de base}

\section{Introduction}

\section{Apprentissage Automatique (Machine Learning)}

Bla

\subsection{Concepts de base}

Bla

\subsection{La régression linéaire}

Bla

\subsection{Forêt aléatoire (Random Forest)}

Bla

\section{Apprentissage Profond (Deep Learning)}

Bla

\subsection{Réseaux de neurones profonds}

Bla

\subsection{Algorithmes spécifiques (e.g., LSTM, CNN)}

Bla

\section{Régression par Ensemble}

Bla

\subsection{Concepts de régression par ensemble}

Bla

\subsection{Avantages des méthodes d'ensemble}

Bla

\subsection{Algorithmes d'ensemble (e.g., Gradient Boosting, XGBoost)}

Bla Bla

%\part*{Contribution}

\chapter{Implémentation des Modèles}

\section{Introduction}

\section{Présentation des Outils Utilisés}

Bla

\subsection{Langages et librairies (e.g., Python, Scikit-learn, TensorFlow)}

Bla

\subsection{Infrastructure matérielle (e.g., GPU, serveurs)}

Bla

\section{Méthodes de l'Apprentissage Automatique}

Bla

\subsection{Le jeu de données}

Bla

\subsection{Préparation et nettoyage des données}

Bla

\subsection{Implémentation des modèles de machine learning}

Bla

\section{Méthodes de l'Apprentissage Profond}

Bla

\subsection{Le jeu de données}

Bla

\subsection{Préparation des données pour le deep learning}

Bla

\subsection{Implémentation des modèles de deep learning}

Bla

\chapter{RÉSULTAT ET DISCUSSIONS}
Ce chapitre se concentre sur l’analyse des résultats de prédiction obtenus après l’application des méthodes de machine learning aux données climatiques et aux cas antérieurs de Chikungunya dans trois pays : le \textbf{Tchad}, le \textbf{Brésil} et le \textbf{Paraguay}. L’objectif est de présenter en détail les résultats, de proposer une discussion approfondie, et d’identifier les éventuelles lacunes ou limitations.

\section{Méthode de Validation Croisée (Entraînement et Test)}
L'ensemble de données, composé de 366 instances pour le \textbf{Tchad} et de 1826 instances pour le \textbf{Brésil} et le \textbf{Paraguay}, comprenant des données climatiques et des informations sur les cas de Chikungunya, a été divisé en deux parties pour l’apprentissage. Tout d’abord, les données ont été mélangées, puis elles ont été séparées en deux ensembles : 80 \% des données ont été utilisées pour l’entraînement (ensemble d’apprentissage) et 20 \% ont été réservées pour les tests (ensemble de test).

\section{Choix des Hyperparamètres des Modèles d'Ensemble}
Les hyperparamètres des modèles utilisés pour déterminer leurs performances sont détaillés ci-dessous.

\subsection{Cas du Random Forest Regressor}
Cette section détaille les choix des hyperparamètres spécifiques pour le \textbf{Random Forest Regressor} et leur impact sur la précision des prédictions. Ces hyperparamètres ont été ajustés à l’aide de la méthode de \textbf{recherche en grille} (\textbf{Grid Search}) afin d’optimiser les performances du modèle illustré dans le tableau \ref{tab:grid-search-rf}.

\begin{table}[!hbt]
	\centering
	\caption{Description des hyperparamètres optimisés pour le Random Forest Regressor avec GridSearch}
	\label{tab:grid-search-rf}
	\begin{tabular}{|c|c|c|c|c|}
		\hline
		& \multicolumn{4}{c|}{Hyperparamètres} \\
		\hline
		Pays & \textsf{n\_estimators} & \textsf{max\_depth} & \textsf{min\_samples\_leaf} & \textsf{min\_samples\_split} \\
		\hline
		Tchad & 200 & 10 & 2 & 2 \\
		\hline
		Brésil & 200 & 20 & 2 & 2 \\
		\hline
		Paraguay & 200 & None & 1 & 5 \\
		\hline
	\end{tabular}
\end{table}

Les hyperparamètres utilisés pour le \textbf{Random Forest Regressor} sont les suivants :
\begin{itemize}
	\item \textbf{n\_estimators} : Le nombre d'arbres dans la forêt. Un nombre plus élevé d'arbres peut améliorer la performance, mais augmente aussi le temps de calcul.
	\item \textbf{max\_depth} : La profondeur maximale des arbres. Une profondeur plus grande permet au modèle de capturer plus de complexité, mais risque de surapprendre (overfitting).
	\item \textbf{min\_samples\_leaf} : Le nombre minimum d'échantillons requis pour être dans une feuille. Un nombre plus élevé peut entraîner un modèle plus simple et réduire le surapprentissage.
	\item \textbf{min\_samples\_split} : Le nombre minimum d'échantillons requis pour diviser un nœud interne. Un nombre plus élevé peut rendre le modèle moins sensible au bruit.
\end{itemize}

\subsection{Cas du XGBoost Regressor}
Cette section présente les hyperparamètres choisis pour le \textbf{XGBoost Regressor} et leur impact sur la précision des prédictions. Ces hyperparamètres ont également été ajustés à l’aide de la méthode de \textbf{recherche en grille} (\textbf{Grid Search}) afin d’optimiser les performances du modèle illustré dans le tableau \ref{tab:grid-search-xb}.

\begin{table}[!hbt]
	\centering
	\caption{Description des hyperparamètres optimisés pour le XGBoost Regressor avec GridSearch}
	\label{tab:grid-search-xb}
	\begin{tabular}{|c|c|c|c|c|}
		\hline
		& \multicolumn{4}{c|}{Hyperparamètres} \\
		\hline
		Pays & \textsf{n\_estimators} & \textsf{max\_depth} & \textsf{subsample} & \textsf{learning\_rate} \\
		\hline
		Tchad & 300 & 5 & 0.8 & 0.2 \\
		\hline
		Brésil & 100 & 5 & 1.0 & 0.2 \\
		\hline
		Paraguay & 100 & 7 & 0.1 & 0.2 \\
		\hline
	\end{tabular}
\end{table}

Les hyperparamètres utilisés pour le \textbf{XGBoost Regressor} sont les suivants :
\begin{itemize}
	\item \textbf{n\_estimators} : Le nombre d'arbres dans la forêt.
	\item \textbf{max\_depth} : La profondeur maximale des arbres. Comme pour le Random Forest, une plus grande profondeur peut capturer plus de complexité, mais risque de surapprentissage.
	\item \textbf{subsample} : La proportion d'échantillons utilisés pour entraîner chaque arbre. Une valeur inférieure à 1.0 réduit le surapprentissage en introduisant plus de diversité dans les arbres.
	\item \textsf{learning\_rate} : Le taux d'apprentissage, qui contrôle la contribution de chaque modèle supplémentaire. Un taux plus bas nécessite plus d'estimators pour converger, mais peut conduire à une meilleure généralisation.
\end{itemize}

\section{Résultats obtenus par nos modèles}
Cette section, nous exposons nos résultats à travers un tableau, suivi d’une discussion approfondie. Par la suite, nous illustrons les diverses mesures de performance à l’aide de graphiques variés.
\subsection{Tableau des performance de nos modèles}
Le tableau~\ref{tab:performance} ci-dessous illustre les métriques d’évaluation des résultats de nos travaux.

Les résultats obtenus montrent que le \textit{modèle d'ensemble}, en particulier le \textbf{Voting Regressor}, a globalement surpassé les autres modèles en termes de précision. Il a permis de réduire les erreurs (\textbf{MAE} et \textbf{RMSE}) et d'augmenter le score R\textsuperscript{2}, indiquant une meilleure explication de la variance des données.

Pour le \textbf{Paraguay}, le \textbf{Voting Regressor} a obtenu un RMSE  faible (\textbf{62.25}) et le meilleur score R\textsuperscript{2} (\textbf{0.5957}), surpassant les modèles LinearRegressor et XGBoostRegressor.

Au \textbf{Brésil}, bien que le modèle XgboostRegressor ait montré de très bonnes performances (RMSE de\textbf{1459.15} et R\textsuperscript{2} de \textbf{0.65}), le \textbf{Voting Regressor} a réussi à obtenir des résultats comparables avec un RMSE particulièrement bas de \textbf{1387.23}.

Pour le \textbf{Tchad}, les performances des modèles étaient globalement plus faibles, mais le \textbf{Voting Regressor} a tout de même montré une légère amélioration par rapport aux autres modèles, bien que le score R\textsuperscript{2} reste faible (\textbf{0.282}).

En conclusion, les modèles d'ensemble, et particulièrement le \textbf{Voting Regressor}, a révélés être les plus performants pour prédire les cas de Chikungunya dans les trois pays étudiés.Toutefois, avec des prédictions plus fiables pour le \textbf{Brésil} et le \textbf{Paraguay} que pour le Tchad, ce qui est dû à la quantité des données disponibles pour ce pays là.

\begin{table}[h!]
	\centering
	\caption{Métriques de performance des modèles (MAE, RMSE, R² Score) pour le Brésil, le Tchad et le Paraguay}
	\begin{tabular}{|c|c|c|c|c|}
		
		\hline
		\textbf{Pays} & \textbf{Modèle} & \textbf{MAE} & \textbf{RMSE} & \textbf{R² Score} \\ 
		\hline
		\multirow{4}{*}{Brésil} & Linear Regression & 1178.99 & 1591.25 & 0.447 \\ 
		& Random Forest Regressor & 897.17  & 1477.20 & 0.523 \\ 
		& XGBoost Regressor & 827.16  & 1459.15 & 0.535 \\ 
		& \textbf{Voting Regressor} & \textbf{840.14}  & \textbf{1387.23} & \textbf{0.65} \\ 
		\hline
		\multirow{4}{*}{Tchad}  & Linear Regression & 56.06   & 98.34   & 0.198 \\ 
		& Random Forest Regressor & 47.81   & 80.90   & 0.457 \\ 
		& XGBoost Regressor & 55.17   & 101.13  & 0.152 \\ 
		& \textbf{Voting Regressor} & \textbf{50.52}   & \textbf{93.02}   & \textbf{0.282} \\ 
		\hline
		\multirow{4}{*}{Paraguay} & Linear Regression & 67.15   & 84.68   & 0.332 \\ 
		& Random Forest Regressor & 35.31   & 61.28   & 0.650 \\ 
		& XGBoost Regressor & 40.32   & 71.95   & 0.517 \\ 
		& \textbf{Voting Regressor} & \textbf{40.37}   & \textbf{62.25}   & \textbf{0.59,97} \\ 
		\hline
	\end{tabular}
	
\end{table}

\subsection{Performance en RMSE , R2 Score et MAE}
La figure \ref{fig:metriccomparaison} montre en termes de RMSE et de coefficient de détermination R2, le niveau de performance de chacun des modèles.
\begin{figure}[h!]
	\centering
	\includegraphics[width=0.8\linewidth]{images/metric_comparaison}
	\caption{Comparaison performance}
	\label{fig:metriccomparaison}
\end{figure}
\newpage
\subsection{Prédiction}
Ci-dessous les prédiction des test dans chaque pays pour le modèle d'ensemble (\textbf{VotingRegressor})
\subsubsection{Pour le Brésil}
La figure \ref{fig:predictionbresil} illustre la phase de prédiction du chikungunya au brésil. 
\begin{figure}[h!]
	\centering
	\includegraphics[width=1\linewidth]{images/prediction_bresil}
	\caption{Prédiction Brésil}
	\label{fig:predictionbresil}
\end{figure}

\subsubsection{Pour le Tchad}
La figure \ref{fig:predictionchad} illustre la phase de prédiction du chikungunya au Tchad. 
\begin{figure}[h!]
	\centering
	\includegraphics[width=0.85\linewidth]{images/prediction_chad}
	\caption{Prédiction Tchad}
	\label{fig:predictionchad}
\end{figure}
\subsubsection{Pour le Paraguay}
La figure \ref{fig:predictionparaguay} illustre la phase de prédiction du chikungunya au paraguay. 
\begin{figure}[h!]
	\centering
	\includegraphics[width=0.85\linewidth]{images/prediction_paraguay}
	\caption{Prédiction Paraguay}
	\label{fig:predictionparaguay}
\end{figure}
\subsection{\textquotedblleft Forecasting\textquotedblright de la maladie au Brésil (3 ans)}
La figure \ref{fig:prediction} illustre la phase de prédiction du chikungunya au brésil. 
\begin{figure}[h!]
	\centering
	\includegraphics[width=0.85\linewidth]{images/pred_chikv}
	\caption{Prédiction Brésil}
	\label{fig:prediction}
\end{figure}
\newpage
\section{Discussion}

Les résultats obtenus dans notre étude montrent une performance compétitive des modèles de régression appliqués aux données climatiques et épidémiologiques de la Chikungunya. En particulier, notre modèle d'ensemble \textit{Voting Regressor}, qui combine les prédictions des modèles \textit{Linear Regressor}, \textit{RandomForest Regressor}, et \textit{XGBoost Regressor}, a affiché des performances globalement supérieures aux autres modèles individuels, avec un \textbf{MAE} minimal et un \textbf{RMSE} relativement bas.

Dans le cas du \textbf{Paraguay}, le \textit{Voting Regressor} a atteint un \textbf{MAE} de \textbf{40.37}, soit une amélioration par rapport au \textit{LinearRegressor}  qui affichent des \textbf{MAE} de \textbf{67.15} . Le RMSE du \textit{Voting Regressor} est de 62.25, indiquant une bonne capacité de prédiction malgré la complexité des données épidémiologiques et climatiques. Le score R² de \textbf{0.59 } montre que ce modèle explique environ 60 \% de la variance des données pour cette région, bien évidement que les techniques de \textbf{data augmentation} et du \textbf{KNNimputer} ont été d'une importance crucial sur ces résultats.

Pour le \textbf{Brésil}, le \textit{Voting Regressor} a montré une robustesse similaire avec un score R² de \textbf{0.65}, ce qui est supérieur à celui du \textit{RandomForestRegressor} et du \textit{XGBoostRegressor} et un \textbf{RMSE} minimale (1387.23) comparé aux autres modèles. Cela suggère que le \textit{Voting Regressor} parvient à combiner les forces de chaque modèle pour fournir des prédictions plus équilibrées.

Au \textbf{Tchad}, bien que les performances générales des modèles soient moins élevées, le \textit{Voting Regressor} reste le modèle le plus performant avec un \textbf{MAE} de \textbf{50.52} et un RMSE de \textbf{93.02}. Le score R² de \textbf{0.282} est le plus élevé parmi les modèles testés pour cette région, indiquant que ce modèle capture mieux la variance des données malgré des difficultés apparentes liées à la la quantité des données disponibles pour cette région, bien évidement que les techniques de \textbf{data augmentation} et du \textbf{KNNimputer} ont été d'une importance crucial sur ces résultats.

Dans notre contexte, l'imprécision observée dans certaines régions, comme au Tchad, pourrait être due à un manque de prédicteurs ou de variables explicatives dans le modèle, ce qui suggère qu'une exploration plus approfondie des facteurs climatiques et épidémiologiques est nécessaire. De plus, l'impact des variables climatiques telles que la \textbf{température} et l'\textbf{humidité} mérite une attention particulière. Il a été observé, par exemple, qu'une augmentation de la température pourrait être associée à une diminution des cas de Chikungunya, tandis que l'\textbf{humidité} semble accroître ces cas. Ce genre de corrélations, bien que contre-intuitives, pourrait fournir des pistes intéressantes pour affiner les modèles prédictifs.

\part*{Conclusion} 
 
\chapter*{Conclusion Générale et perspectives}
\addcontentsline{toc}{chapter}{Conclusion Générale et Perspectives}

L'objectif principal de cette étude était de développer un modèle prédictif du chikungunya, en utilisant des approches de régression d'ensemble issues de l'intelligence artificielle en examinant l'influence des variables climatiques sur la propagation du Chikungunya au Brésil, au Paraguay et au Tchad, en s'appuyant sur des modèles de \textbf{régression avancés}. 

Les données épidémiologiques ont été obtenues sur le site du monitoring des cas d'épidomologie en temps réel PAHO (pour le Paraguay) ,dans le rapport d'OMS(pour le cas du Tchad) et dans le site mendeley pour celui du brésil tandis que les données climatiques provenaient de sources fiables telles que le site WeatherAndClimate.  Les modèles choisis pour cette étude incluaient le \textbf{Random Forest Regressor} et le \textbf{XGBoost Regressor} optimisé via \textbf{Grid Search}, ainsi qu'un modèle d'ensemble (\textbf{Voting Regressor}) combinant \textbf{Linear Regression}, \textbf{Random Forest Regressor} et le \textbf{XGBoost Regressor} optimisé. Parmi ces modèles, notre modèle d'ensemble \textbf{Voting Regressor}, qui combine les prédictions des modèles \textbf{Linear Regression}, \textbf{Random Forest Regressor} et \textbf{XGBoost Regressor}, a affiché des performances globalement supérieures aux autres modèles individuels, avec un \textbf{MAE} minimal, un \textbf{RMSE} relativement bas et une très bonne précision (\textbf{91,08\%} pour le Brésil, \textbf{34,34\%} pour le Tchad et \textbf{59,57\%} pour le Paraguay).

Les limites de l’étude ont montré que les variables climatiques ne suffisent pas à elles seules à expliquer les variations des cas de chikungunya dans ces pays. Il est proposé, pour les futures
recherches, d’intégrer d’autres facteurs environnementaux et de développer un modèle hybride combinant des algorithmes d’apprentissage automatique pour affiner les prévisions. Les recommandations incluent l’amélioration de la collecte des données, l’adoption de stratégies adaptées aux variations climatiques, et la sensibilisation des communautés sur l’hygiène et la prévention du chikungunya.









% récupérer les citations avec "/footnotemark"
\nocite{*}

% choix du style de la biblio
%\bibliographystyle{plain}
%\addbibresource{bibliographie}
\bibliographystyle{spmpsci}
% inclusion de la biblio
\bibliography{bibliographie}
% voir wiki pour plus d'information sur la syntaxe des entrées d'une bibliographie

%\input{./chapters/annexes.tex}

\newpage
\end{spacing}{}

\end{document}
